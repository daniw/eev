\documentclass[a4,paper,fleqn]{article}

\usepackage{layout}

\DeclareSIUnit\year{Jahr}
\newcommand{\wye}{
    \begin{tikzpicture}
        \draw ( 90:0) -- ( 90:0.2);
        \draw (210:0) -- (210:0.2);
        \draw (330:0) -- (330:0.2);
    \end{tikzpicture}
}

\title{Notizen EEV -- SW08}
\date{\today}
\author{Daniel Winz}

\begin{document}
\maketitle
\clearpage

\section{Sternpunktbehandlung}
(siehe Unterlagen)

\subsection{Erdfehlerfaktor}
\[ f_E = \frac{U_E}{\frac{U_N}{\sqrt{3}}} \]
$U_E$: Spannung zwischen einem "'gesunden"' Aussenleiter und der Erde bei 
einem 1pES. 
\subsubsection{wirksame Erdung}
\[ f_E \text{ ist überall } < 1.4 \]
\subsubsection{unwirksame Erdung}
\[ f_E > 1.4 \]

\subsection{Betriebskapazität}
Handnotiz
\[ C': \text{ Beträge in F pro km und Phase} \]
Stern-Dreieck-Transformation
\[ X_{C_L} = \frac{1}{\omega \cdot C_L} \rightarrow \frac{X_{C_L}}{3} 
= \frac{1}{3 \cdot \omega \cdot C_L} \]
\[ \Rightarrow {C_{L_\star}}' \Rightarrow 3 \cdot {C_L}' \]
Handnotiz\\
Parallelschaltung: 
\[ \boxed{{C_b}' = {C_E}' + 3 \cdot {C_L}'} \]

\subsection{Netze mit isoliertem Sternpunkt}
Mittelspannungsnetze mit geringer Ausdehnung
\[ \Rightarrow C_b \text{ klein} \]
Handnotiz
\[ I_{E_2} = U_{12} \cdot j \cdot B_{CE} \qquad (B_{CE} = \omega \cdot C_E) \]
\[ I_{E_3} = U_{31} \cdot j \cdot B_{CE} \]
\[ I_E = I_{E_2} - I_{E_3} \qquad \text{(Knoten)} \]
Weil $U_{E_2}$ und $U_{E_3}$ 120$^\circ$ phasenverschoben sind, sind es auch $I_{E_2}$ und $I_{E_3}$
\[ \boxed{\left|I_{E_2} - I_{E_3}\right| = \sqrt{3} U_N \omega C_E = 3 I_0 = I_E} \]
\[ I_0 = U_{q_1} \omega C_E \]
\[ I_E = 3 I_0 = 3 U_{q_1} \omega C_E = 3 \frac{U_N}{\sqrt{3}} \omega C_E \]
\[ I_E = \sqrt{3} U_N \omega C_E \]
Reduktion von $I_E$ durch "'Löschspule"' bzw. Petersenspule im Sternpunkt 
deines Trafos. 
\\Nullsystem: 
\\ Handnotiz
\[ X_T << 3 X_D \qquad \text{Nullreaktanz des Trafos vernachlässigen} \]
$\Rightarrow$ Parallelschwingkreis\\
Resonanz bei $y = 0$
Handnotiz
\[ y_D = \frac{B_D}{3} \qquad \text{aus } \frac{1}{3 X_D} = \frac{1}{3\omega L} \]
\[ B_D = \frac{1}{\omega L_D} \]
Bedingung: 
\[ \frac{B_D}{3} = \omega C_E \]
\[ \Rightarrow \boxed{L_D = \frac{1}{3 \omega^2 C_E}} \qquad \Rightarrow I_E = 0 \]
$L_D$ muss dem Netzzustand angepasst werden. \\
Wirkwiderstände führen zu einer Verstimmung des Schwingkreises. \\
Fehlabstimmungen liegen im Vereich $\pm 20\%$. \\
Bemerkung: $C_E$ ist die Erdkapazität des gesamten Netzes. 

\end{document}
