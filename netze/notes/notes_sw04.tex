\documentclass[a4,paper,fleqn]{article}

\usepackage{layout}

\DeclareSIUnit\year{Jahr}
\newcommand{\wye}{
    \begin{tikzpicture}
        \draw ( 90:0) -- ( 90:0.2);
        \draw (210:0) -- (210:0.2);
        \draw (330:0) -- (330:0.2);
    \end{tikzpicture}
}

\title{Notizen EEV -- SW04}
\date{\today}
\author{Daniel Winz}

\begin{document}
\maketitle
\clearpage

\section{Erneuerbare Energie}

\subsection{Potential}
\begin{itemize}
    \item theoretisch: Angebot der Natur
    \item technisch: Erschliessungsmöglichkeiten
    \item realisierbar: wirtschaftliche Nutzung
\end{itemize}

\[ \text{Erntefaktor} = 
    \frac{\text{elelktrische Produktion eines Kraftwerks}}
         {\text{Energie zur Herstellung}} \]
\[ \text{Energetische Amortisationszeit} = 
    \frac{\text{Energie zur Herstellung}}
         {\text{elektrische Jahresproduktion}} \]

\section{Leistung einer Windanlage}
\begin{tabular}{@{}ll}
    $D$:        & Durchmesser des Rotors \\
    $\varrho$:  & Dichte der Luft \\
    $v$:        & Windgeschwindigkeit \\
    $\eta$:     & Wirkungsgrad \\
\end{tabular}
\\
Kinetische Energie eines Luftzylinders mit Masse $m$ und Geschwindigkeit $v$
\[ W_{kin} = \frac{1}{2} m v^2 \]
\[ m = v \cdot \varrho = A \cdot \ell \cdot \varrho \]
\[ A = frac{\pi \cdot D^2}{4} \]
\[ \ell = v cdot t \]
\[ \Rightarrow m = \frac{\pi D^2}{4} v \ell \varrho \]
\[ \Rightarrow W_{kin} = \frac{1}{8} \pi D^2 v^3 t \varrho \]
\[ \Rightarrow P_{Wind} = \frac{W_{kin}}{t} = \frac{1}{8} \pi D^2 v^3 \varrho \]

\end{document}
