\documentclass[a4,paper,fleqn]{article}

\usepackage{layout}

\DeclareSIUnit\year{Jahr}
\newcommand{\wye}{
    \begin{tikzpicture}
        \draw ( 90:0) -- ( 90:0.2);
        \draw (210:0) -- (210:0.2);
        \draw (330:0) -- (330:0.2);
    \end{tikzpicture}
}

\title{Notizen EEV -- SW05}
\date{\today}
\author{Daniel Winz}

\begin{document}
\maketitle
\clearpage

\section{Symmetrische Impedanzen}
Bestimmbar bei zyklisch, symmetrischem Aufbau der Betriebsmittel. 
\\
\begin{tabular}{ll}
Mitimpedanz   & $\underline{Z}_1$ \\
Gegenimpedanz & $\underline{Z}_2$ \\
Nullimpedanz  & $\underline{Z}_0$ \\
\end{tabular}
\\
Es gilt: 
\[ \underline{U}_1 = \underline{Z}_1 \cdot \underline{I}_1 \qquad \text{Normalbetrieb} \]
\[ \underline{U}_2 = \underline{Z}_2 \cdot \underline{I}_2 \]
\[ \underline{U}_0 = \underline{Z}_0 \cdot \underline{I}_0 \]

\subsection{Messung der symmetrischen Impedanzen}
(Längsimpedanzen für Kurzschlussrechnungen)
\[ \boxed{ \underline{Z}_1} \qquad \text{Normalbetrieb} \]
\begin{circuitikz}
    \draw
        % N
        (0,0)
            to[short, *-]
        (0, 2)
            to[V<=$U_{1N}$, i=$I_{L1}$]
        (2, 2)
            to[short]
        (4, 2)
        (0,0)
            node[left] {$N$} to[short, *-]
        (0, 0)
            to[V<=$U_{2N}$, i=$I_{L2}$]
        (2, 0)
            to[short]
        (4, 0)
        (0,0)                  to[short, *-]
        (0,-2)
            to[V<=$U_{3N}$, i=$I_{L3}$]
        (2,-2)
            to[short]
        (4,-2)
        % K
        (8, 2)
            to[short]
        (10, 2)
            to[short, -*]
        (10,0)
        (8, 0)
            to[short]
        (10, 0)
            to[short, -*]
        (10,0)
            node[right] {$K$}
        (8,-2)
            to[short]
        (10,-2)
            to[short, -*]
        (10,0)
    ;
    \draw (4,-3) rectangle node[align=center] {Betriebsmittel\\$\underline{Z}_1$} (8,3);
1
\end{circuitikz}
\[ \underline{Z}_1 = \frac{\underline{U}_{1N}}{\underline{I}_{L1}} \]

\[ \boxed{ \underline{Z}_2} \]
Bemerkung: $\underline{Z}_2 = \underline{Z}_1$ für nichtrotierende Betriebsmittel. 
\\
\begin{circuitikz}
    \draw
        % N
        (0,0)
            to[short, *-]
        (0, 2)
            to[V<=$U_{1N}$, i=$I_{L1}$]
        (2, 2)
            to[short]
        (4, 2)
        (0,0)
            node[left] {$N$} to[short, *-]
        (0, 0)
            to[V<=$U_{2N}$, i=$I_{L2}$]
        (2, 0)
            to[short]
        (4,-2)
        (0,0)                  to[short, *-]
        (0,-2)
            to[V<=$U_{3N}$, i=$I_{L3}$]
        (2,-2)
            to[short]
        (4, 0)
        % K
        (8, 2)
            to[short]
        (10, 2)
            to[short, -*]
        (10,0)
        (8, 0)
            to[short]
        (10, 0)
            to[short, -*]
        (10,0)
            node[right] {$K$}
        (8,-2)
            to[short]
        (10,-2)
            to[short, -*]
        (10,0)
    ;
    \draw (4,-3) rectangle node[align=center] {Betriebsmittel\\$\underline{Z}_2$} (8,3);
2
\end{circuitikz}
\[ \underline{Z}_2 = \frac{\underline{U}_{1N}}{\underline{I}_{L1}} \]

\[ \boxed{ \underline{Z}_0} \]
\begin{circuitikz}
    \draw
        % N
        (0,-2)
            to[short, *-]
        (0, 2)
            to[short, i=$I_{0}$]
        (2, 2)
        (0, 0)
            node[left] {$N$} to[short, *-]
        (0, 0)
            to[short]           
        (2, 0)
        (0,-2)                  to[short, *-]
        (0,-2)
            to[short]           
        (2,-2)
        % K
        (6, 2)
            to[short]
        (8, 2)
            to[short, -*]
        (8,-2)
        (6, 0)
            to[short]
        (8, 0)
            to[short, -*]
        (8, 0)
            node[right] {$K$}
        (6,-2)
            to[short]
        (8,-2)
            to[short, -*]
        (8,-2)
        % ZE
        (0,-2)
            to[V_=$\underline{U}_0$, *-]
        (0,-6)
            to[short]
        (3,-6)
            to[generic=$Z_E$, v<=$U_E$]
        (5,-6)
            to[short]
        (8,-6)
            to[short, i<_=$3 \cdot \underline{I}_0$]
        (8,-2)
    ;
    \draw (2,-3) rectangle node[align=center] {Betriebsmittel\\$\underline{Z}_{0B}$} (6,3);
    \draw[-latex] (2,-3.2) -- node[below] {$\underline{U}_{0B}$} (6,-3.2);
3
\end{circuitikz}
\[ \underline{Z}_0 
= \frac{\underline{U}_{0B} + \underline{U}_{E}}{\underline{I}_{0}}
= \frac{\underline{U}_{0}}{\underline{I}_{0}}
= \frac{Z_{0B} \cdot \underline{I}_{0} + Z_{E} \cdot 3 \underline{I}_{0}}{\underline{I}_{0}}
= Z_{0B} + 3 \cdot Z_{E} \]

\subsection{Ersatzschaltungen}
Mitsystem: \\
\begin{circuitikz}
    \draw
        (0,0)
            to[short, -o]
        (3,0)
            to[open, v<=$\underline{U}_1$, o-o]
        (3,2)
            to[generic, l_=$\underline{Z}_1$, i<=$\underline{I}_1$, o-]
        (0,2)
            to[V_=$\underline{U}_{q1}$]
        (0,0)
    ;
4
\end{circuitikz} \\
Gegensystem: \\
\begin{circuitikz}
    \draw
        (0,0)
            to[short, -o]
        (3,0)
            to[open, v<=$\underline{U}_2$, o-o]
        (3,2)
            to[generic, l_=$\underline{Z}_2$, i<=$\underline{I}_2$, o-]
        (0,2)
            to[V_=$\underline{U}_{q2}$]
        (0,0)
    ;
5
\end{circuitikz} \\
Nullsystem: \\
\begin{circuitikz}
    \draw
        (0,0)
            to[short, -o]
        (3,0)
            to[open, v<=$\underline{U}_0$, o-o]
        (3,2)
            to[generic, l_=$\underline{Z}_0$, i<=$\underline{I}_0$, o-]
        (0,2)
            to[V_=$\underline{U}_{q0}$]
        (0,0)
    ;
6
\end{circuitikz} \\

\subsection{Vereinfachungen für Kurzschlussrechnungen}
\begin{itemize}
    \item Starre Speisung: 
        \[ \underline{U}_{q1} = \underline{U}_{1N} \]
        \[ \underline{U}_{q2} = \underline{U}_{q0} = 0 \]
    \item Es dominieren die Reaktanzen \\
        (X) der Selbst- und Gegeninduktivitäten
    \item Bildung einer Ersatzquelle
        (Thévenin) am Ort des Fehlers
\end{itemize}
Definition: \\
\begin{tabular}{@{}ll}
Generatornaher Fehler:  & vor Trafo \\
Generatorferner Fehler: & nach Trafo
\end{tabular}

\subsection{Impedanzen am Trafo}
\[ Z_{OS} = \frac{{U_{OS}}^2}{S} \]
\[ Z_{US} = \frac{{U_{US}}^2}{S} \]

\subsection{Generator}
\begin{tikzpicture}
7
\end{tikzpicture}
\[ \underline{\underline{X_I = X_d'' = \kappa_d'' \cdot \frac{{U_N}^2}{S_N}}} \]
${X_d}''$: subttransiente Reaktanz
\[ {X_d}'' \approx \frac{X_d}{6} \]
Für $U_N$ wird die \emph{Spannung am Kurzschlussort} eingesetzt! 
\[ X_2 = X_1    \qquad \text{für Vollpolmaschinen} \]
\[ X_2 > X_1    \qquad \text{für Schenkelpolmaschinen} \]
\[ X_0 = \infty \qquad \text{keine Erdung des Sternpunktes (Stator)} \]

\subsection{Netzeinspeisung (Ersatznetz)}
\begin{tikzpicture}
8
\end{tikzpicture}
\[ {S_k}'' = \sqrt{3} \cdot U_N \cdot {I_k}'' \]
\[ \underline{\underline{X_1 = X_2 = c \cdot \frac{{U_0}^2}{{S_k}''}}} \]
$c$: Spannungsfaktor \\
NE1 und NE3: $c = 1.1$ (Höhere Spannung in Praxis)
\[ X_0 = \infty \qquad \text{$Z_E$ sehr gross} \]

\subsection{Transformator}
\begin{tikzpicture}
9
\end{tikzpicture}
$u_K$: relative Kurzschlussspannung
\[ u_K = \frac{U_K}{U_{1N}} \]
$U_K$: Spannung die angelegt werden muss um $I_N$ zu erhalten beim Kurzschluss
\[ X_0 = X_1 \qquad \text{für Dreischenkelkern mit geerdetem Sternpunkt} \]

\subsection{Freileitungen}
Induktivitätsbelag
\[ L' \left(\frac{\text{H}}{\text{km Phase}}\right) \]
\[ \omega \cdot L' = {X_L}' \left(\frac{\text{$\Omega$}}{\text{km Phase}}\right) \]
\[ X_1 = X_2 = {X_L}' \cdot \ell \]
$\ell$: Länge in km
\[ X_0 = 3 \cdot X_I \qquad \text{Näherung für Freileitung mit Erdseil} \]

\end{document}
