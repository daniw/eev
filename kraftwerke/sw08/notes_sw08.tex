\documentclass[a4,paper,fleqn]{article}

\usepackage{layout}

\DeclareSIUnit\year{Jahr}

\title{Notizen EEV -- SW08}
\date{\today}
\author{Daniel Winz}

\begin{document}
\maketitle
\clearpage

\section{Thermodynamik}

\subsection{1. Hauptsatz}
Energie kann weder "'vernichtet"' noch "'erzeugt"' werden. \\
Wärme = Energieform (Wirkungsgrad, Verluste)

\subsection{2. Hauptsatz}
\begin{tikzpicture}
    \draw (0,0) rectangle node {$T_1$} (2,4);
    \draw (2,0) rectangle node {$T_2$} (4,4);
    \draw[-latex] (1,3) -- node[above] {$Q$} (3,3);
\end{tikzpicture}
\\
$Q$ = Wärmefluss
\[ T_1 > T_2 \]
Prinzip thermische Maschine
\\Handnotiz

\subsection{Aggregats-Zustände z.B. Wasser}
Handnotiz
\begin{tikzpicture}
    \draw[-latex] (0,0) -- (5,0) node[below] {Energie $W$};
    \draw[-latex] (0,0) -- (0,5) node[left] {Temperatur $\vartheta$};
    \draw (0,0) -- (1,1) -- (2,1) -- (4,3) -- (5,3) -- (6,4);
\end{tikzpicture}

\subsection{Nassdampf}
Handnotiz
1 kg Nassdampf\\
$\Rightarrow$ x kg Sattdampf, (1-x) kg siedende Flüssigkeit\\
Siedelinie $\Rightarrow$ x=0\\
Taulinie $\Rightarrow$ x=1
\[ x = \frac{\text{Masse Dampf}}{\text{Masse Wasser + Masse Dampf}} = \frac{m''}{m' + m''} \]
Handnotiz

\subsection{Zustandsgrössen ideale Gase}
Isothermiegesetz $T=const$
\[ \frac{p_1}{p_2} = \frac{V_2}{V_1} \]
Isobarengesetz $p=const$
\[ \frac{T_1}{T_2} = \frac{V_1}{V_2} \]
\[ \text{Wärmemenge:} Q = m \cdot c_p \cdot (T_2 - T_1) \]
Isochore Zustandsänderung $V=const$
\[ \frac{T_1}{T_2} = \frac{p_1}{p_2} \]
\[ \text{Wärmemenge:} Q = m \cdot c_v \cdot (T_2 - T_1) \]
Gasgleichung
\[ p \cdot V = m \cdot n_s \cdot T \]
\begin{tabular}{@{}l@{$=$}l}
    $p  $ & Druck \\
    $V  $ & Volumen \\
    $m  $ & Masse \\
    $n_s$ & Gaskonstante \\
    $T  $ & Temperatur [K] \\
\end{tabular}

\section{Prinzip Dampf-Kraftwerk}
Handnotiz

\subsection{Dampf-Kraftwerk Kreislauf Skizze}
Handnotiz

\subsection{Brennstoffe}
\begin{itemize}
    \item Kohle (weltweit hoher Anteil)
    \item Erdgas (eher selten für Dampf-Kraftwerk)
    \item Uran
    \item Erdöl (Schweröl)
    \item Biomasse (gewinnt an Bedeutung, $CO_2$ neutral)
\end{itemize}

\end{document}
